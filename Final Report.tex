% Final Report Template
\documentclass[12pt]{article}
\usepackage[utf8]{inputenc}
\usepackage{graphicx}
\usepackage{hyperref}

% Title and Author
\title{Final Report: Seismology Class Project}
\author{Erdem}
\date{\today}

\begin{document}

\maketitle

\tableofcontents
\newpage

% Abstract Section
\section*{Abstract}
\addcontentsline{toc}{section}{Abstract}
This report summarizes the work done for the Seismology class project. The repository contains scripts and data related to synthetic seismogram generation, waveform processing, and earthquake analysis.

% Introduction Section
\section{Introduction}
Provide an overview of the project, its goals, and the significance of the work.

% Methodology Section
\section{Methodology}
Describe the methods and tools used in the project. Include details about the scripts and their purposes:
\begin{itemize}
    \item \texttt{calcMLsingle.py}: Explain its functionality.
    \item \texttt{convert_mseed_to_sac.py}: Describe its role in data conversion.
    \item \texttt{create_station0lines.py}: Mention its purpose.
    \item Other scripts: Briefly summarize their contributions.
\end{itemize}

% Results Section
\section{Results}
Present the findings or outputs of the project. Include any relevant figures, tables, or data visualizations.

% Discussion Section
\section{Discussion}
Interpret the results and discuss their implications. Highlight any challenges faced during the project.

% Conclusion Section
\section{Conclusion}
Summarize the key takeaways from the project and suggest potential future work.

% References Section
\section*{References}
\addcontentsline{toc}{section}{References}
List any references or resources used in the project.

\end{document}